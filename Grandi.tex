\documentclass{article}

\usepackage[utf8]{inputenc}

\usepackage{amsmath}
\usepackage{amssymb}

\title{S = 1 - S
~\\
doesn't resolve to
~\\
-1/12
}
\date{
~\\
2019-12-16
}
\author{
~\\
Damiens ROBERT
}
\pagenumbering{arabic}

\begin{document}
  \maketitle
  \section{Grandi's series}
  \paragraph{}
    As everybody knows, the Grandi's series is demonstrated by subtracting 2 said identical series.
    \begin{itemize}
      \item
  	    $S$
      \item
  	    $1 - S$
    \end{itemize}
    Let's show why those 2 series are actually different.
  \section{S is a 1 dimension matrix}
    \subsection{Transform a serie in a 1 dimension matrix}
      ~\\
      \begin{equation*}
        S = 1 - 1 + 1 - 1 + ... + 1
      \end{equation*}
      ~\\
      or
      ~\\
      \begin{equation*}
        S = 1 - 1 + 1 - 1 + ... - 1
      \end{equation*}
      ~\\
      We can transform S into a 1 dimension matrix using this method :
      \begin{itemize}
        \item
          Using only addition in the series by replacing the substraction of the number $n_i$ by $+ (-n_i)$ where i is the position in the serie.
        \item
	  Put each term of the serie of $T$ terms in a 1 dimension matrix of size $T$
      \end{itemize}

      For the serie $S$, we then get :
      ~\\
      \begin{equation*}
        S = (1, -1, ..., 1)
      \end{equation*}
      ~\\
      or
      ~\\
      \begin{equation*}
        S = (1, -1, ..., -1)
      \end{equation*}
      ~\\
    \subsection{0 as a 1 dimension matrix ?}
      0 depends on the the serie it is being added.
    \subsection{justification}
      In order to be able to add 2 matrices, the matrices needs to have the same size.
    \subsection{Example with S}
      ~\\
      \centerline{
      S = (1, -1, ... , 1)
      }
      ~\\
      \centerline{
      S = (0,  0, ... , 0)
      }
      ~\\
      or
      ~\\
      \centerline{
      S = (1, -1, ... , -1)
      }
      ~\\
      \centerline{
      S = (0,  0, ... , 0)
      }
    \section{Conclustion}
      We cannot substract $S$ from 1 because 1 is a natural number and $S$ is a matrix.
      ~\\
      ~\\
      What is 1 as a matrix ? All 1's ? Only one 1 ? Multiple 1's ?
      ~\\
      ~\\
      The 1 from natural number encoded as a matrix of size N is this matrix :
      ~\\
      ~\\
      \centerline{
      1 = (1,  1, ... , 1)
      }
      ~\\
      or this one :
      ~\\
      \centerline{
      1 = (1,  0, ... , 0)
      }
      ~\\
      or yet another one ?
      ~\\
      ~\\
      What is sure, is that 1 - S won't be the same dependending on your definition of 1 represented as a matrix and this ambiguity is what leads to a mistake in the Grandi's series demonstration.
\end{document}
