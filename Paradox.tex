\documentclass{article}

\usepackage[utf8]{inputenc}

\usepackage{array}

\usepackage{amsmath}
\usepackage{amssymb}

\title{0 is not neutral in a serie
~\\
when
~\\
S in a Grandi's series
~\\
is
~\\
a matrix
}
\date{
~\\
2019-12-16
}
\author{
~\\
Damiens ROBERT
}
\pagenumbering{arabic}

\begin{document}
  \maketitle
  \section{Introduction}
    In this text, we leave the notation of 0 intentionally ambiguous. I will fix this in a following text.
    \paragraph{}
      As everybody knows, the Grandi's series is demonstrated by the equality.
      \begin{equation*}
      1 - S = S
      \end{equation*}
      ~\\
      Some mathematicians clame that it converges to $1/2$.
      ~\\
  \section{Computing the wrong way}
    ~\\
    \centerline{
    1 - S = S
    }
    ~\\
    where S = 1 - 1 + 1 - 1 + ...
    ~\\
    ~\\
    Wrong step is following !
    ~\\
    \centerline{
    1 - S = 1 - (1 - 1 + 1 - 1 + 1 + ...)
    }
    ~\\
    Note that the wrong step is used in the Grandi series demonstration in order to prove $S=1/2$.
  \section{Computing the right way}
    ~\\
    Let
    ~\\
    \centerline{
    1 = (1, 0, 0, 0, ..., 0) where ... is all 0's
    }
    ~\\
    Then make the correct computation for 1 - S by subtracting 2 matrices.
    ~\\
    ~\\
    Then notice that S $\neq$ 1 - S
  \section{Another wrong way ...}
    We could then think we just need to subtract 2 matrices of the same size.
    ~\\
    \centerline{
    S + 0 = (1, -1, 1, -1, ... , -1 , 0)
    }
    ~\\
    or  
    ~\\
    \centerline{
    0 + S = (0, 1, - 1, 1, ...)
    }
    ~\\
    Note that if we add those two matrices we get.
    ~\\
    ~\\
    \centerline{
    0 + S - S + 0 $\neq$ 0
    }
    ~\\
    which would prove adding 0 is not a commutative operation when dealing with infinite series.
    ~\\
    ~\\
    It would also prove that 0 is not the a neutral when dealing with infinite series.
    \subsection{What it truly is !}
      \subsubsection{1 - S produces the complement vector}
        A matrix is also a vector in geometry. When using multiple dimensions in a geographical space, the Natural number $1$ is
        ~\\
        ~\\
        \centerline{
        1 = (1, 0, 0, 0, ..., 0) where ... is all zero's
        }
        ~\\
        then, if we say that 1 - S is subtracting S to the Natural number 1,
        ~\\
        ~\\
        \centerline{
        (1 - S) = (0, -1, 1, -1, ... ) where ... is all 1 or -1 one after another 
        }
        ~\\
      \subsubsection{Vector arithmethics resolves correctly}
        You can then project this vector on the same origin than the vector (S) and find out the vector +(1 - S) is -(S).
        ~\\
        ~\\
        +(1 - S) + -(S) = 0 becomes -(S) + (S) = 0 which resolves to (0) = 0.
      \subsubsection{Algebra also resolves correctly}
        ~\\
        Let
        ~\\
        ~\\
        \centerline{
        (0) + +(S) + -(S) = 0
        }
        ~\\
        becomes
        ~\\
        ~\\
        \centerline{
        (0, ..., 0) + (1, -1, 1, -1, ...) - (-1, 1, -1, 1, ...) = 0
        }
        ~\\
        ~\\
        You will be able to make yourself this formal proof after reading the texte where I prove thaty (+$\infty$) + ($-\infty$) = 0.
        ~\\
        ~\\
        Let's remind that (-S) = (1 - S).
  \section{Conclusion}
    S = S, 1 - S = -S, S - S = 0 and 0 = 0 when resolving 1 - S = S properly.
    ~\\
    ~\\
    I will provide another more formal proof in a following text but I first need to introduce notations. This will be done in the (+$\infty$) + ($-\infty$) = 0 article.
\end{document}
