\documentclass{article}

\usepackage[utf8]{inputenc}

\usepackage{amsmath}
\usepackage{amssymb}

\title{Using A clever operation
~\\
in order to
~\\
have
~\\
The best possible encryption algorithm
~\\
for
~\\
a given message
}
\date{01-08-2020
~\\
}
\author{
~\\
Damiens ROBERT
}
\pagenumbering{arabic}

\begin{document}
  \maketitle
  \section{Demonstration}
    Let the message M be a matrix whose elements are either 1, either 0.
    ~\\
    Let the encrypted message $\dot{M}$ be a matrix whose elements are either 1, either 0.
    ~\\
    ~\\
    Let's find the relation R that project M onto $\dot{M}$ and the relation $\dot{R}$ that projects $\dot{M}$ onto M.
    ~\\
    ~\\
    Let's first notice that we have every possible answer for a given element in M and $\dot{M}$ is a set of couples which are the element in M and the element in $\dot{M}$.
    ~\\
    ~\\
    \centerline{
    ((0,0) , (0,1) , (1,0) , (1,1))
    }
    ~\\
    Let's notice that if the element in M and $\dot{M}$, are either identical or different.
    ~\\
    ~\\
    The relation R must build the possible couples in order to be able to construct $\dot{M}$ is the XOR operator. We then obtain the set of element in E where E is the encryption key :
    ~\\
    ~\\
    \centerline{
    (0 , 1 , 1 , 0)
    }
    ~\\
    Let's notice that the relation to obtain M from $\dot{M}$ and E is also a XOR.
\end{document}
