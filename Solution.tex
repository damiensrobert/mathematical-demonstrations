\documentclass{article}

\usepackage[utf8]{inputenc}

\usepackage{amsmath}
\usepackage{amssymb}

\title{
}
\date{01-07-2020
~\\
}
\author{
~\\
Damiens ROBERT
}
\pagenumbering{arabic}

\begin{document}
  \maketitle
  \section{}
  Let $\omega$ be the universal set.
  ~\\
  Let $\dot{\omega}$ be also the universal set.
  ~\\
  Let $\epsilon$ be a subset of $\omega$ which is a metalanguage.
  ~\\
  Let $\dot{\epsilon}$ be a subset od $\dot{\omega}$ which is the equipotent set of $\epsilon$ projected from $\omega$ in $\dot{\omega}$.
  ~\\
  ~\\
  Let's prove that there exists a set such as $\epsilon$.
  ~\\
  In order to do so, we need to find a bijective relation between 2 different sets and the bijective relation must be true for any given subset of $\omega$.
  ~\\
  Such a relation exists for numbers. This relation consist of adding a number to itself.
  ~\\
  $\epsilon$ is then any set of numbers and $\dot{\epsilon} is the set of same cardinality than $\espilon$ where any number k in $\epsion$ is k+k in $\dot{\epsion}.
\end{document}
